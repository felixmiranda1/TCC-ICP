\chapter{TRABALHOS RELACIONADOS}

Foram considerados como trabalhos relacionados aqueles que abordam técnicas de \textit{machine learning} aplicadas ao \textit{lead scoring} e à definição do \textit{Ideal Customer Profile} (ICP), incluindo modelos de classificação de uma classe (\textit{One-Class Classification} – OCC), métodos baseados em distância (\textit{Distance-Based Scoring} – DBS) e abordagens híbridas de segmentação. O objetivo desta seção é compreender como diferentes técnicas têm sido aplicadas em contextos semelhantes, bem como evidenciar lacunas que justificam a proposta desenvolvida no presente trabalho.


No campo da classificação de uma classe, \citet{Seliya2021} apresentam uma revisão abrangente das técnicas de One-Class Classification (OCC), destacando sua aplicabilidade em cenários onde a disponibilidade de dados rotulados negativos é limitada ou inexistente. Eles enfatizam que métodos OCC são particularmente úteis para detecção de anomalias e identificação de perfis específicos, o que é diretamente relevante para a definição do ICP em ambientes de \textit{lead scoring}. A abordagem teórica e prática discutida por Seliya et al. fornece uma base sólida para a aplicação desses modelos em contextos comerciais, onde a segmentação precisa de clientes potenciais é crucial.

Complementando essa perspectiva, \citet{Wu2023} exploram modelos avançados de \textit{lead scoring}, integrando técnicas supervisionadas e não supervisionadas para melhorar a precisão na identificação de leads qualificados. Sua análise destaca a importância de incorporar características comportamentais e demográficas, além de considerar a escassez de dados negativos, o que reforça a utilidade dos métodos OCC. A pesquisa de Wu et al. demonstra como a combinação de diferentes abordagens pode superar limitações tradicionais, alinhando-se com a proposta deste trabalho que busca integrar múltiplas técnicas para aprimorar a definição do ICP.

Por fim, \citet{Nygard2020} investigam casos práticos de automação no \textit{lead scoring}, evidenciando ganhos significativos em eficiência e precisão ao aplicar algoritmos de aprendizado de máquina em processos comerciais. Seu estudo de caso mostra como a implementação de modelos automatizados pode transformar a gestão de leads, reduzindo o esforço manual e aumentando a taxa de conversão. Essa experiência empírica reforça a relevância da automação inteligente, um aspecto central da presente pesquisa, que visa desenvolver uma solução robusta e escalável para a segmentação e priorização de leads utilizando técnicas de OCC e métodos híbridos.

Complementarmente, \citet{Qian2019} apresentam uma abordagem baseada em modelos de distância para o ranqueamento de entidades, demonstrando que medidas de similaridade podem ser aplicadas de maneira eficaz em contextos de priorização. Sua pesquisa evidencia como técnicas de \textit{distance-based scoring} oferecem maior flexibilidade na comparação entre instâncias, especialmente quando combinadas com atributos heterogêneos. Essa perspectiva contribui para este trabalho ao fundamentar a utilização de métricas de distância como mecanismo de apoio à classificação e hierarquização de leads.

Na mesma linha de integração entre técnicas, \citet{Mancisidor2018} investigam a aplicação de autoencoders em conjunto com classificadores tradicionais, visando aprimorar a segmentação de dados complexos. O estudo mostra como representações latentes extraídas por redes neurais podem potencializar a etapa de classificação, resultando em melhorias no desempenho preditivo. Essa estratégia dialoga diretamente com a proposta deste TCC, que busca explorar arquiteturas híbridas capazes de unir a robustez de modelos OCC com métodos de ranqueamento baseados em distância.

Por outro lado, \citet{Golbayani2020} realizam um estudo comparativo sobre a previsão de ratings corporativos, confrontando o desempenho de Redes Neurais, Máquinas de Vetores de Suporte (SVM) e Árvores de Decisão. Seus resultados indicam que não há um modelo universalmente superior, mas que a eficácia depende do contexto e da qualidade dos dados utilizados. Essa constatação reforça a importância de adotar uma estratégia híbrida, conforme delineado neste trabalho, que combina diferentes paradigmas de modelagem para lidar com a variabilidade dos dados de empresas e otimizar a identificação do ICP.

De forma conjunta, os trabalhos analisados evidenciam a diversidade de estratégias aplicáveis à definição de perfis ideais de clientes e ao \textit{lead scoring}, variando entre revisões teóricas, estudos de caso práticos e experimentos comparativos de modelos. A integração dessas contribuições ressalta que não existe uma solução única e definitiva, mas sim a necessidade de combinar técnicas de forma criteriosa. Essa constatação fundamenta a proposta central deste trabalho, que adota uma estratégia híbrida entre OCC e DBS para superar limitações individuais e oferecer uma abordagem mais robusta e adaptável à identificação do ICP em empresas fornecedoras de benefícios corporativos.