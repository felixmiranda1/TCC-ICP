\chapter{INTRODUÇÃO}

O ambiente corporativo contemporâneo é caracterizado por elevada competitividade e por ciclos de vendas cada vez mais complexos, especialmente em negócios do tipo B2B (business-to-business). Nesse contexto, cresce a necessidade de identificar com precisão quais clientes representam maior potencial de retorno, reduzindo esforços comerciais e maximizando resultados.

O conceito de Ideal Customer Profile (ICP) surge como resposta a essa demanda, oferecendo um método estruturado para compreender quais empresas apresentam maior alinhamento com a proposta de valor da organização (PONO, 2020). Mais do que uma ferramenta de segmentação, o ICP se consolida como um instrumento estratégico que orienta decisões de marketing, priorização de leads e planejamento comercial (EXPERIAN, 2020).

A aplicação de modelos computacionais voltados à previsão do ICP representa um avanço relevante para o setor B2B, pois permite decisões baseadas em dados e não apenas em julgamento humano. No segmento de benefícios corporativos — exemplificado por empresas como Unimed, Swile, TotalPass, Gympass e Psicologia Viva —, essa abordagem é especialmente promissora, dado o alto custo e a complexidade das negociações. Assim, compreender e modelar o ICP torna-se essencial para aprimorar a eficiência das estratégias de prospecção e conversão.

\section{OBJETIVOS}

Considerando a crescente competitividade nos mercados B2B (business-to-business), em especial no setor de benefícios corporativos, e a necessidade das empresas em otimizar seus processos de prospecção e qualificação de clientes, este trabalho tem como objetivo desenvolver um modelo computacional capaz de apoiar a identificação do Ideal Customer Profile (ICP). A proposta busca integrar diferentes fontes de dados firmográficos e contextuais, explorando técnicas de aprendizado não supervisionado e de ranqueamento por similaridade, de forma a contribuir para maior assertividade na priorização de leads, redução de custos no processo comercial e suporte a estratégias de marketing orientadas por dados. 

São objetivos secundários: 

\begin{itemize}
    \item Estruturar uma pipeline de aquisição e enriquecimento de dados firmográficos, integrando informações provenientes de diferentes fontes digitais;
    \item Realizar o pré-processamento dos dados, incluindo limpeza, padronização, imputação de valores faltantes e vetorização das variáveis categóricas e contínuas;
    \item Implementar e avaliar técnicas de One-Class Classification (OCC) para identificar empresas não aderentes ao perfil desejado;
    \item Aplicar métricas de Distance-Based Scoring (DBS) para ranquear as empresas remanescentes de acordo com sua proximidade ao ICP;
    \item Comparar os resultados obtidos com modelos supervisionados de referência, como regressão logística, discutindo vantagens e limitações;
    \item Analisar o potencial de aplicação prática do modelo no setor de benefícios corporativos, destacando seus impactos em eficiência comercial e priorização de leads.
\end{itemize}

\section{ORGANIZAÇÃO}

Este trabalho está estruturado em seis capítulos, além dos elementos pré-textuais e pós-textuais exigidos pelas normas acadêmicas.

No Capítulo 1, apresenta-se a introdução, contemplando o contexto e a motivação do estudo, a formulação do problema de pesquisa, os objetivos geral e específicos e a organização geral do documento.  

O Capítulo 2 aborda os fundamentos teóricos que embasam o trabalho, incluindo o conceito de Ideal Customer Profile (ICP), sua importância no funil de vendas em negócios B2B, as principais variáveis firmográficas utilizadas nesse processo, além de uma revisão sobre técnicas de machine learning relevantes, como One-Class Classification (OCC) e Distance-Based Scoring (DBS). Também são discutidos trabalhos relacionados que exploram metodologias semelhantes no contexto de priorização de clientes.  

O Capítulo 3 descreve o processo de aquisição e tratamento de dados, detalhando as estratégias de coleta utilizadas, como web scraping, consumo de APIs públicas e normalização via CNPJ, bem como os procedimentos de limpeza, enriquecimento e preparação da base final para análise.  

O Capítulo 4 apresenta a construção do modelo computacional, contemplando as etapas de pré-processamento, a implementação da camada OCC para filtragem de empresas não aderentes ao ICP, a aplicação do DBS para ranqueamento e a definição do fluxo híbrido proposto.  

No Capítulo 5 são discutidos os experimentos computacionais, nos quais os modelos são aplicados à base de dados construída. São apresentados os resultados da filtragem por OCC, do ranqueamento por DBS, da comparação com modelos supervisionados de referência e da análise crítica dos impactos práticos no setor de benefícios corporativos.  

Por fim, o Capítulo 6 traz as conclusões e trabalhos futuros, destacando as principais contribuições alcançadas, as limitações identificadas e as perspectivas de evolução da metodologia, incluindo a possibilidade de integração com sistemas corporativos de CRM e a aplicação de técnicas mais avançadas de aprendizado de máquina.