\chapter{FUNDAMENTAÇÃO TEÓRICA}

Neste capítulo, serão apresentados os conceitos essenciais para a compreensão do trabalho, bem como as técnicas utilizadas em sua proposta metodológica. Inicialmente, será discutido o conceito de Ideal Customer Profile (ICP) e sua relevância em estratégias de marketing e vendas no contexto B2B (business-to-business), destacando seu papel dentro do funil de vendas. Em seguida, aborda-se a importância dos dados firmográficos e das fontes de informação corporativa para a caracterização de empresas e a formação de bases consistentes. Posteriormente, são introduzidos os principais modelos de classificação aplicados ao ICP, com ênfase em técnicas de One-Class Classification (OCC), voltadas para a detecção de empresas não aderentes ao perfil ideal, e de Distance-Based Scoring (DBS), responsáveis pelo ranqueamento das empresas de acordo com sua proximidade ao ICP.

\section{IDEAL CUSTOMER PROFILE (ICP) E MARKETING B2B}

O Ideal Customer Profile (ICP) é definido como a representação sistemática das características do cliente que oferece maior potencial de sucesso e rentabilidade para uma organização (PONO, 2020). Trata-se de uma ferramenta amplamente empregada em estratégias de marketing e vendas B2B, permitindo direcionar recursos de forma mais eficiente e reduzir custos de aquisição (EXPERIAN, 2020).

Segundo Inflexion-Point Strategy Partners (2020), a elaboração de um ICP envolve a análise de variáveis firmográficas — como setor de atuação, porte, localização e tecnologias empregadas —, além de padrões de comportamento e necessidades específicas. Essa estrutura auxilia não apenas na identificação de leads com maior probabilidade de conversão, mas também na exclusão de perfis pouco aderentes.

A literatura ressalta que o ICP atua como guia ao longo de todo o funil de vendas: na prospecção, ajuda a priorizar contatos qualificados; na qualificação, oferece critérios objetivos de viabilidade; na conversão, aumenta a taxa de fechamento; e na retenção, contribui para o aumento do valor do ciclo de vida do cliente (LTV) e a redução do custo de aquisição (CAC) (McKINSEY \& COMPANY, 2020; TOPO, 2020).

Em um cenário de concorrência acirrada e tomadas de decisão cada vez mais orientadas por dados, o ICP assume papel central nas estratégias data-driven, promovendo previsibilidade comercial e sustentando o alinhamento entre oferta e demanda — aspecto particularmente relevante para empresas do setor de benefícios corporativos, cujos ciclos de venda são longos e de alta complexidade.
\section{DADOS FIRMOGRÁFICOS E FONTES DE DADOS CORPORATIVOS}

Para a caracterização de empresas no contexto de definição do Ideal Customer Profile (ICP), o uso de dados firmográficos representa um dos pilares fundamentais. Analogamente aos dados demográficos, utilizados para descrever indivíduos, os dados firmográficos descrevem atributos estruturais e contextuais de organizações, permitindo sua categorização e comparação. Entre os exemplos mais comuns encontram-se o porte da empresa, o número de funcionários, o capital social, o segmento de atuação (CNAE/indústria) e a localização geográfica (EXPERIAN, 2020).
Essas variáveis desempenham papel estratégico na priorização de leads, uma vez que permitem identificar clientes com maior aderência ao ICP, além de excluir empresas fora do escopo de interesse. Por exemplo, fornecedores de benefícios corporativos tendem a focar em organizações de médio e grande porte, com capital social elevado e alta concentração de colaboradores em determinadas regiões, de modo a maximizar o impacto da oferta. Assim, a análise firmográfica possibilita a construção de critérios objetivos de qualificação que complementam a experiência das equipes de vendas (PONO, 2020).
A obtenção desses dados pode ocorrer por diferentes meios. No contexto brasileiro, destacam-se fontes como a ReceitaWS e a BrasilAPI, que oferecem informações vinculadas ao Cadastro Nacional da Pessoa Jurídica (CNPJ), incluindo razão social, porte, capital social e atividade econômica principal. Complementarmente, o Instituto Brasileiro de Geografia e Estatística (IBGE) disponibiliza tabelas oficiais de Classificação Nacional de Atividades Econômicas (CNAE), fundamentais para padronizar a identificação de segmentos. Além disso, dados coletados em plataformas digitais — como LinkedIn e sistemas de divulgação de vagas de emprego — permitem enriquecer a análise com informações sobre contratações, funções desempenhadas e setores em expansão.
Apesar de sua importância, o uso de dados firmográficos apresenta desafios significativos. A heterogeneidade de formatos entre diferentes fontes, a existência de valores ausentes ou desatualizados e a necessidade de normalização representam barreiras que exigem processamento criterioso. Outro ponto crucial é a atenção à Lei Geral de Proteção de Dados (LGPD), que impõe cuidados éticos e legais na coleta e no tratamento de informações, ainda que de natureza corporativa. Dessa forma, a etapa de aquisição e tratamento de dados deve ser cuidadosamente projetada para garantir a confiabilidade e a integridade das informações utilizadas no modelo.

\section{MODELOS DE MACHINE LEARNING NÃO SUPERVISIONADO APLICADOS À IDENTIFICAÇÃO DO ICP}

O presente trabalho utiliza técnicas da área de Aprendizado de Máquina (Machine Learning), um campo da Inteligência Artificial que busca desenvolver algoritmos capazes de extrair padrões a partir de dados, possibilitando a tomada de decisões ou a realização de predições sem a necessidade de regras programadas manualmente. Dentre as várias categorias existentes no Aprendizado de Máquina, os métodos adotados neste estudo pertencem à classe dos algoritmos de aprendizado não supervisionado, isto é, algoritmos que aprendem a estrutura dos dados sem contar com rótulos pré-definidos que indiquem a categoria ou o valor esperado para cada instância.

Essa abordagem é especialmente adequada para o contexto deste projeto, pois a base de dados utilizada é composta por empresas que são clientes ativas de fornecedores de benefícios corporativos, como Gympass e TotalPass. No entanto, apesar de todas essas empresas fazerem parte da carteira de clientes dessas organizações, não há uma anotação explícita indicando quais delas realmente representam o perfil ideal (Ideal Customer Profile — ICP) e quais foram adquiridas de maneira eventual, fora do padrão estratégico da empresa. Por esse motivo, optou-se por técnicas capazes de identificar anomalias dentro do conjunto de dados, bem como ranquear os elementos com base em sua similaridade ao grupo principal.

A modelagem proposta combina dois grupos de algoritmos de aprendizado não supervisionado: os modelos de detecção de anomalias e os modelos baseados em distância. Os primeiros, conhecidos como One-Class Classification (OCC), são treinados apenas com exemplos considerados “normais” e aprendem uma fronteira que os separa de observações anômalas. Esses modelos são frequentemente utilizados em cenários onde apenas exemplos positivos estão disponíveis, como em detecção de fraudes, análise de falhas e perfis de clientes. 

Por sua vez, os modelos baseados em distância não constroem uma fronteira de decisão, mas avaliam o quanto cada observação se aproxima de uma referência construída com base no conjunto de dados — como o centróide, representado pela média vetorial, ou os vizinhos mais próximos, como no método k-Nearest Neighbors. Esses métodos são úteis para gerar um score contínuo de aderência ao perfil médio observado, permitindo o ranqueamento das empresas de acordo com sua compatibilidade com o ICP.

\subsection{One-Class Classification (OCC) e Tratamento de Outliers}

O \textit{One-Class Classification} (OCC) é uma abordagem utilizada em cenários
nos quais apenas uma classe de interesse está disponível, e o objetivo é
identificar instâncias que se desviam significativamente desse padrão
(\textit{outliers}). Em vez de distinguir entre múltiplas categorias,
o OCC busca modelar a distribuição dos exemplos considerados ``normais'',
definindo uma fronteira que engloba a região de maior densidade dos dados
e rejeita observações fora dessa região. 

No contexto deste trabalho, o OCC é aplicado à identificação do
\textit{Ideal Customer Profile} (ICP), permitindo modelar diretamente a
distribuição das empresas com características típicas do perfil ideal e
filtrar aquelas que se afastam substancialmente desse padrão. Assume-se,
de forma plausível, que a maior parte das empresas da base analisada
representa clientes adequados ao ICP, enquanto uma minoria corresponde a
casos atípicos ou menos representativos.

\subsubsection{\textit{Isolation Forest (IF)}}

Entre as técnicas de OCC, o \textit{Isolation Forest} destaca-se por sua
eficiência e simplicidade conceitual. O método baseia-se na ideia de que
instâncias anômalas são mais fáceis de isolar por meio de particionamentos
aleatórios do espaço de atributos. Constrói-se uma floresta de árvores de
isolamento, em que cada nó divide os dados escolhendo aleatoriamente um
atributo e um ponto de corte. O número médio de divisões necessárias para
isolar uma instância $x$ define seu \textit{comprimento de caminho}
$h(x)$: observações normais tendem a exigir mais quebras, enquanto
outliers são isolados rapidamente. O \textit{score} de anomalia é
calculado como:

\begin{equation}
s(x, n) = 2^{-\frac{E[h(x)]}{c(n)}}, \qquad
c(n) = 2H_{n-1} - \frac{2(n - 1)}{n},
\end{equation}

onde $H_k$ é o $k$-ésimo número harmônico e $c(n)$ atua como fator de
normalização do caminho esperado.

\subsubsection{\textit{Aplicação ao ICP e filtragem de outliers}}

Na presente pesquisa, o \textit{Isolation Forest} foi utilizado como
etapa preliminar de filtragem antes da aplicação do modelo
\textit{Distance-Based Scoring} (DBS). Essa escolha se justifica pela
natureza não supervisionada do problema: ainda que todas as empresas da
base sejam clientes ativas de organizações como TotalPass, Gympass ou
Swile, é razoável supor que nem todas representem o perfil ideal.
Algumas podem ter sido adquiridas por abordagens comerciais pontuais,
pertencer a segmentos secundários ou apresentar características
distantes do foco estratégico atual.

A aplicação do \textit{Isolation Forest} permitiu excluir, de forma
estatisticamente fundamentada, essas observações discrepantes. O modelo
foi configurado com uma taxa de contaminação de 5\%, assumindo que
aproximadamente essa fração das empresas poderia ser considerada
anômala. Essa parametrização segue recomendações da literatura para
problemas de detecção de anomalias em bases de clientes B2B, onde a
maioria das instâncias é presumidamente legítima.

O procedimento foi conduzido sobre os dados já vetorizados e escalados,
garantindo que as métricas de separabilidade considerassem o conjunto
completo de variáveis firmográficas e contextuais. Apenas as instâncias
marcadas como \textit{inliers} pelo \textit{Isolation Forest} foram
mantidas para a etapa subsequente de ranqueamento via DBS, assegurando
que as medidas de proximidade fossem calculadas sobre um subconjunto
representativo e coerente com o padrão estatístico dominante.

Essa integração entre OCC e DBS resultou em um processo mais robusto,
capaz de combinar o rigor matemático da detecção de anomalias com a
interpretação de negócio necessária à identificação do ICP. O filtro
prévio de outliers reduziu o ruído, aumentou a consistência dos escores
de proximidade e aprimorou a confiabilidade da inferência sobre o perfil
ideal de cliente.




\subsection{\textbf{Distance-Based Scoring (DBS)}}

O \textit{Distance-Based Scoring} (DBS) é uma abordagem que consiste em
atribuir um score contínuo a cada instância com base em sua proximidade
a um ponto de referência representativo da classe de interesse. No
contexto de ICP, esse ponto de referência pode ser entendido como uma
representação central das empresas consideradas clientes ideais, de modo
que organizações mais próximas a esse centro recebem escores mais altos
de similaridade, enquanto aquelas mais distantes recebem escores mais
baixos.

\subsubsection{\textit{Métrica Euclidiana e centróide ICP}}

Seja $X_{\text{in}}=\{x_i\}_{i=1}^{n}$ o conjunto de vetores de atributos dos
\textit{inliers}. O centróide ICP é definido como a média aritmética:
\begin{equation}
\mu \;=\; \frac{1}{n}\sum_{i=1}^{n} x_i\, .
\end{equation}

A proximidade de uma empresa $x$ ao perfil ideal é medida pela distância
euclidiana ao centróide:
\begin{equation}
d_E(x,\mu)\;=\;\sqrt{\sum_{j=1}^{d}(x_j-\mu_j)^2}\, .
\end{equation}

Para interpretar proximidade como \textit{score} (maior é melhor),
normalizou-se as distâncias com \textit{Min–Max} ao intervalo $[0,1]$ e
invertêmo-las:
\begin{equation}
s_{\text{cent}}(x)\;=\;1-\frac{d_E(x,\mu)-\min\limits_{z\in X_{\text{in}}} d_E(z,\mu)}
{\max\limits_{z\in X_{\text{in}}} d_E(z,\mu)-\min\limits_{z\in X_{\text{in}}} d_E(z,\mu)}\, .
\end{equation}


\subsubsection{\textit{Proximidade local por $k$-vizinhos (entre inliers)}}

Complementarmente, avaliamos a densidade local com $k$-vizinhos mais
próximos (\textit{k-NN}) \emph{no conjunto de inliers}, usando distância
euclidiana (padrão do \texttt{NearestNeighbors}, Minkowski $p{=}2$). Para cada
$x\in X_{\text{in}}$, computa-se a distância média aos $k$ vizinhos (excluindo
o próprio ponto):
\begin{equation}
\bar{d}_{k}(x)\;=\;\frac{1}{k}\sum_{i=1}^{k} d_E\!\big(x,x_{(i)}\big)\, ,
\end{equation}
em que $x_{(i)}$ denota o $i$-ésimo vizinho mais próximo de $x$ em
$X_{\text{in}}$. O respectivo \textit{score} é obtido por normalização
\textit{Min–Max} e inversão:
\begin{equation}
s_{k\text{NN}}(x)\;=\;1-\frac{\bar{d}_{k}(x)-\min\limits_{z\in X_{\text{in}}} \bar{d}_{k}(z)}
{\max\limits_{z\in X_{\text{in}}} \bar{d}_{k}(z)-\min\limits_{z\in X_{\text{in}}} \bar{d}_{k}(z)}\, .
\end{equation}

\subsubsection{\textit{Considerações gerais}}

As métricas baseadas em distância permitem construir um \textit{ranking
contínuo} de aderência ao ICP, complementando a filtragem inicial
realizada por técnicas como o OCC. Sua principal vantagem é fornecer
granularidade: em vez de apenas classificar instâncias como dentro ou
fora do perfil, o DBS ordena as empresas de acordo com seu grau relativo
de similaridade. Por outro lado, essas técnicas podem ser sensíveis à
escolha da métrica e à escala dos atributos, exigindo normalização
adequada e, em alguns casos, ponderação diferenciada entre blocos de
variáveis.


\subsection{\textbf{Abordagem Híbrida OCC + DBS}}

Embora técnicas de \textit{One-Class Classification} (OCC) e \textit{Distance-Based
Scoring} (DBS) possam ser aplicadas de forma independente, a combinação
de ambas se mostra particularmente adequada em cenários de identificação
de ICP, nos quais há escassez de rótulos explícitos e alta
heterogeneidade dos dados disponíveis. A abordagem híbrida consiste em
aplicar o OCC como uma etapa inicial de filtragem, removendo instâncias
com baixa probabilidade de pertencerem ao perfil ideal, seguido pelo
DBS, responsável por atribuir um score contínuo de similaridade às
instâncias remanescentes.

\subsubsection{\textit{Estrutura do fluxo híbrido}}

O fluxo pode ser descrito em três etapas principais:  
1. \textbf{Filtragem inicial (OCC):} empresas consideradas muito discrepantes em relação ao
conjunto ICP são classificadas como outliers e eliminadas.  
2. \textbf{Cálculo de sscores (DBS):} para as empresas restantes, calcula-se
a proximidade em relação a um centro representativo do ICP, atribuindo
sscores contínuos de similaridade.  
3. \textbf{Ranqueamento final:} as empresas são ordenadas de acordo com o
escore, possibilitando a priorização de leads de maior aderência.

\subsubsection{\textit{Vantagens da abordagem híbrida}}

A combinação OCC + DBS une duas propriedades complementares:  
- O OCC fornece robustez contra ruído e instâncias atípicas, garantindo
que apenas dados plausíveis sigam adiante.  
- O DBS introduz granularidade, estabelecendo níveis de proximidade que
permitem ordenar candidatos de acordo com sua relevância.

Assim, em vez de uma classificação binária (ICP vs. não-ICP), obtém-se
um espectro contínuo de similaridade, mais adequado a contextos de
tomada de decisão em vendas e marketing B2B.

\subsubsection{\textit{Considerações finais}}

A adoção do fluxo híbrido permite reduzir significativamente a
subjetividade na construção do ICP, fornecendo um processo reprodutível,
auditável e orientado por dados. Além disso, a metodologia é flexível:
diferentes variantes de OCC (como Isolation Forest) e métricas
de DBS (como euclidiana) podem ser combinadas conforme as
características do conjunto de dados.
