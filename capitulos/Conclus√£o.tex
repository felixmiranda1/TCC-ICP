\chapter{CONCLUSÃO}
\label{chap:conclusao}

O presente trabalho teve como propósito central o desenvolvimento de um modelo computacional capaz de identificar o \textit{Ideal Customer Profile} (ICP) em contextos B2B, com ênfase no setor de benefícios corporativos. A partir desse objetivo, buscou-se estruturar um processo completo de tratamento, análise e modelagem de dados que pudesse ser reproduzido e aplicado em bases firmográficas reais, fornecendo um instrumento confiável para apoiar a priorização de empresas com maior probabilidade de aderência ao perfil ideal de cliente.

O pipeline proposto foi implementado integralmente em Python, no ambiente Google Colab, e dividido em etapas claramente definidas. Inicialmente, realizou-se o pré-processamento das variáveis firmográficas, contemplando padronização de dados numéricos e codificação de variáveis categóricas por meio de técnicas de \textit{one-hot encoding}. Essa estrutura garantiu a vetorização adequada dos atributos, permitindo a construção de um espaço numérico compatível com algoritmos de aprendizado de máquina e análise de similaridade.

Na sequência, aplicou-se o método de detecção de anomalias \textit{Isolation Forest}, pertencente à família dos modelos de \textit{One-Class Classification} (OCC). Essa etapa não teve o objetivo de compor o ranking final, mas sim de atuar como um filtro de consistência, removendo observações atípicas que poderiam distorcer as métricas de distância subsequentes. A decisão de usar o OCC como etapa de higienização, e não como parte da métrica de priorização, mostrou-se fundamental para preservar a estabilidade estatística do modelo e evitar penalizações indevidas.

Após o processo de limpeza, foram implementadas duas medidas complementares de similaridade por meio do \textit{Distance-Based Scoring} (DBS). A primeira correspondeu à distância euclidiana em relação ao centróide das empresas consideradas válidas, representando o grau de proximidade global de uma empresa ao perfil médio do ICP. A segunda baseou-se na distância média aos dez vizinhos mais próximos (\textit{k}-NN), responsável por capturar a densidade local e as nuances do agrupamento de empresas similares. Ambas as medidas foram normalizadas entre 0 e 1, permitindo a combinação ponderada em um índice final.

A etapa de ranqueamento consolidou essas duas medidas em um único escore híbrido, calculado a partir de uma média ponderada que atribuiu peso de 0,8 à distância ao centróide e 0,2 à distância média aos vizinhos. Essa configuração prioriza empresas que não apenas se aproximam do perfil central definido, mas também se inserem em regiões de maior densidade de observações semelhantes, favorecendo a robustez do ranking. As observações classificadas como \textit{outliers} pelo \textit{Isolation Forest} foram excluídas dessa média, reforçando o caráter seletivo do modelo.

Os resultados obtidos demonstraram coerência e interpretabilidade, com a consolidação de um ranking capaz de distinguir grupos de empresas de forma transparente. Além disso, a análise descritiva e geográfica realizada no Capítulo 6 revelou uma concentração expressiva de empresas na região Sudeste, com destaque para o estado de São Paulo, seguido por Minas Gerais, Paraná, Santa Catarina e Rio de Janeiro. Esse padrão reflete a centralização econômica do setor e oferece evidências objetivas para orientar estratégias comerciais, indicando onde esforços de prospecção e expansão podem ser mais produtivos.

Do ponto de vista técnico, o modelo contribuiu ao demonstrar que uma arquitetura simples, baseada em métodos interpretáveis e com baixa dependência de dados rotulados, pode produzir resultados consistentes e acionáveis em ambientes de negócios. O pipeline modular, a padronização dos hiperparâmetros e a documentação detalhada das etapas conferem reprodutibilidade e transparência ao processo, facilitando futuras expansões e integração com sistemas corporativos.

Por outro lado, algumas limitações merecem destaque. A qualidade do ranking depende diretamente da completude e da padronização das variáveis firmográficas disponíveis, o que pode introduzir vieses em amostras pequenas ou heterogêneas. Além disso, os valores de hiperparâmetros, como a taxa de contaminação do \textit{Isolation Forest} e o número de vizinhos considerados no cálculo da distância média, foram definidos empiricamente e podem exigir ajustes para outros conjuntos de dados. A ausência de rótulos supervisionados também impõe restrições à validação direta da performance do ranking, que, nesta fase, expressa similaridade estrutural ao ICP e não necessariamente probabilidade de conversão.

Mesmo diante dessas limitações, o trabalho se mostrou eficaz ao alcançar seu objetivo central: construir um mecanismo prático, transparente e replicável de identificação do perfil ideal de cliente. A combinação sequencial entre o filtro OCC e o ranking DBS provou-se robusta, estável e de fácil interpretação, constituindo uma alternativa acessível a modelos complexos de aprendizado supervisionado. O uso de métricas de distância, aliado a análises geográficas, permitiu compreender tanto a estrutura global das bases quanto padrões regionais de concentração e dispersão, fortalecendo a aplicabilidade dos resultados em contextos reais de prospecção.

Para trabalhos futuros, recomenda-se ampliar a base de atributos com variáveis comportamentais e de engajamento comercial, integrar o pipeline a sistemas de CRM para retroalimentação contínua e incorporar mecanismos de calibração supervisionada, quando rótulos de conversão estiverem disponíveis. Sugere-se ainda o monitoramento de deriva temporal e o uso de técnicas de explicabilidade (\textit{XAI}) para aprimorar a transparência das recomendações. Dessa forma, o modelo poderá evoluir de uma ferramenta analítica de priorização para um sistema dinâmico de apoio à decisão, com impacto direto sobre a eficiência e a previsibilidade do processo comercial.

Em síntese, este trabalho apresenta uma abordagem híbrida consistente, sustentada em princípios de interpretabilidade e replicabilidade, capaz de oferecer à área de Engenharia Computacional uma aplicação concreta e de relevância prática. Ao integrar fundamentos de modelagem estatística, aprendizado não supervisionado e análise espacial, o estudo reforça o papel da engenharia de dados como instrumento estratégico de tomada de decisão e contribui para o avanço de metodologias de ranqueamento inteligente em ambientes corporativos B2B.